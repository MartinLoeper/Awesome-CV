%!TEX TS-program = xelatex -> biber -> xelatex
%!TEX encoding = UTF-8 Unicode
% Awesome CV LaTeX Template for CV/Resume
%
% This template has been downloaded from:
% https://github.com/posquit0/Awesome-CV
%
% Author:
% Claud D. Park <posquit0.bj@gmail.com>
% http://www.posquit0.com
%
% Template license:
% CC BY-SA 4.0 (https://creativecommons.org/licenses/by-sa/4.0/)
%


%-------------------------------------------------------------------------------
% CONFIGURATIONS
%-------------------------------------------------------------------------------
% A4 paper size by default, use 'letterpaper' for US letter
\documentclass[11pt, a4paper]{awesome-cv}
 
% Configure page margins with geometry
\geometry{left=1.4cm, top=.8cm, right=1.4cm, bottom=1.8cm, footskip=.5cm}

% Specify the location of the included fonts
\fontdir[fonts/]

% Color for highlights
% Awesome Colors: awesome-emerald, awesome-skyblue, awesome-red, awesome-pink, awesome-orange
%                 awesome-nephritis, awesome-concrete, awesome-darknight
\colorlet{awesome}{awesome-red}
% Uncomment if you would like to specify your own color
% \definecolor{awesome}{HTML}{CA63A8}

% Colors for text
% Uncomment if you would like to specify your own color
% \definecolor{darktext}{HTML}{414141}
% \definecolor{text}{HTML}{333333}
% \definecolor{graytext}{HTML}{5D5D5D}
% \definecolor{lighttext}{HTML}{999999}

% Set false if you don't want to highlight section with awesome color
\setbool{acvSectionColorHighlight}{true}

% If you would like to change the social information separator from a pipe (|) to something else
\renewcommand{\acvHeaderSocialSep}{\quad\textbar\quad}


%-------------------------------------------------------------------------------
%	PERSONAL INFORMATION
%	Comment any of the lines below if they are not required
%-------------------------------------------------------------------------------
% Available options: circle|rectangle,edge/noedge,left/right

\ifdefined\isusversion
  % leave the image out because of US anti-discrimination laws
\else
  \photo[circle, edge, left]{./imgs/profile_pic-min.jpg}
\fi

\hypersetup{pdfinfo={
  Title=CV - Martin Loeper,
  Author=Martin Loeper,
  pdflang=en-US,
  Subject=Curriculum Vitae,
  Keywords={cv,martin,loeper,kit,stanford,stromberg,gymnasium,karlsruhe,cybersecurity,devops,cloud,aws,development,software}}
}
 
\name{Martin}{Löper}
\position{Cloud Solutions Architect}
\address{D-75177 Pforzheim, Germany}

\email{martin.loeper@gmx.de}
%\ifdefined\isusversion\else
  \homepage{www.mloeper.me}
%\fi
\linkedin{martinloeper}
\stackoverflow{10473469}{martin-löper}
%\ifdefined\isusversion\else
%  \medium{martin.loeper}
%\fi
%\googlescholar{mj2VByAAAAAJ}{Martin Löper}
%\github{MartinLoeper}
%\ifdefined\isusversion\else
%  \gitlab{MartinLoeper}
%\fi 
% \extrainfo{extra informations}

\quote{``Somewhere between Dev and Ops"}
\addbibresource{./references.bib}

%-------------------------------------------------------------------------------
\begin{document}

% Print the header with above personal informations
% Give optional argument to change alignment(C: center, L: left, R: right)
\makecvheader

% Print the footer with 3 arguments(<left>, <center>, <right>)
% Leave any of these blank if they are not needed
\makecvfooter
  {\today}
  {Martin Löper~~~·~~~Curriculum Vitae}
  {\thepage}


%-------------------------------------------------------------------------------
%	CV/RESUME CONTENT
%	Each section is imported separately, open each file in turn to modify content
%-------------------------------------------------------------------------------
%-------------------------------------------------------------------------------
%	SECTION TITLE
%-------------------------------------------------------------------------------
\cvsection{Education}


%-------------------------------------------------------------------------------
%	CONTENT
%-------------------------------------------------------------------------------
\begin{cventries}

%---------------------------------------------------------
  \cventry
    {Cybersecurity Graduate Certificate: cs 251, cs 212, cs 155 (3.94 GPA as of spring 2021/22)} % Degree
    {\href{https://online.stanford.edu/programs/cybersecurity-graduate-program}{Stanford University (SCPD)} } % Institution
    {Online} % Location
    {Sept. 2021 - PRESENT} % Date(s)
    {}

  \cventry
    {B.Sc. in Information Engineering and Management (3.2 GPA; top 10\% as per ECTS grading table)} % Degree
    {\href{https://www.sle.kit.edu/english/vorstudium/bachelor-information-engineering-management.php}{Karlsruhe Institute of Technology (KIT)} } % Institution
    {Karlsruhe, Germany} % Location
    {Okt. 2014 - Sept. 2020} % Date(s)
    {}
  
  \cventry
    {Abitur / General Higher Education Entrance Qualification  (3.9 GPA; top 3\% statewide)} % Degree
    {\href{https://www.stromberg-gymnasium.de/?page_id=113}{Stromberg-Gymnasium}} % Institution
    {Vaihingen an der Enz, Germany} % Location
    {Sept. 2006 - Jun. 2014} % Date(s)
    {
      \begin{cvitems} % Description(s) bullet points
        \item {Organized the Arduino microcontroller and the homepage group for students.}
      \end{cvitems}
    }

%---------------------------------------------------------
\end{cventries}

%-------------------------------------------------------------------------------
%	SECTION TITLE
%-------------------------------------------------------------------------------
\cvsection{Skills}


%-------------------------------------------------------------------------------
%	CONTENT
%-------------------------------------------------------------------------------
\begin{cvskills}

%---------------------------------------------------------
  \cvskill
  {Industry Knowledge}
  {Web Development, AWS Cloud Architecting, Linux System Administration, Linux \& IoT Systems Programming}  

  \cvskill
    {Development} % Category
    {Node.js, Nest.js, REST, GraphQL, Debian Packaging, Git, AWS SDK, MySQL} % Skills

%---------------------------------------------------------
  \cvskill
    {Operations} % Category
    {DataDog, Sentry, Amazon CloudWatch, Amazon Athena, MongoDB Atlas, AWS CLI} % Skills

%---------------------------------------------------------
  \cvskill
    {DevOps} % Category
    {Docker, Packer, Ansible, GitLab CI, GitHub Actions, AWS CDK, AWS CloudFormation} % Skills

%---------------------------------------------------------
  \cvskill
    {Programming} % Category
    {Typescript, Python, Java, Bash, C++} % Skills

%---------------------------------------------------------
  \cvskill
    {Languages} % Category
    {German (native), Polish (native), English (C1)} % Skills

%---------------------------------------------------------
\end{cvskills}

%-------------------------------------------------------------------------------
%	SECTION TITLE
%-------------------------------------------------------------------------------
\cvsection{Experience}


%-------------------------------------------------------------------------------
%	CONTENT
%-------------------------------------------------------------------------------
\begin{cventries}

%---------------------------------------------------------
\cventry
{Cloud Solutions Architect} % Job title
{Nesto Software GmbH} % Organization
{Karlsruhe, Germany} % Location
{Sept. 2017 - PRESENT} % Date(s)
{
  \begin{cvitems} % Description(s) of tasks/responsibilities
    \item {Built Nesto's cloud infrastructure from the ground up with a strong preference for state-of-the-art AWS architectures and services.}
    \item {Coordinated IoT development, IoT fleet operations, and systems programming - including on-premise installations for clients across Germany.}
    \item {Operated internal services as well as mission-critical databases and SaaS workloads in the cloud.}
    \item {Participated in full-stack application development using Node.js, TypeScript, OpenUI5, and Vue.js.}
    \item {Initially focused on CI/CD automation and classic DevOps tasks.}
  \end{cvitems}
}

%---------------------------------------------------------
\cventry
{Software Developer [working student]} % Job title
{FZI Research Center for Information Technology} % Organization
{Karlsruhe, Germany} % Location
{Apr. 2016 - Jan. 2018} % Date(s)
{
  \begin{cvitems} % Description(s) of tasks/responsibilities
    \item {Operated and developed an internal Java EE time tracking application for the Information Process Engineering department.}
    \item {Connected the application to the LDAP server to foster centralized authentication.}
    \item {Worked with Apache Maven, MySQL, Hibernate ORM, Java Server Pages, Tomcat, Jenkins, Subversion, and Red Hat Enterprise Linux.}
    \item {Gave a talk about software development in small teams using Jira in front of the FZI student colloquium.}
  \end{cvitems}
}

%---------------------------------------------------------
\cventry
{Software Integration Engineer [internship]} % Job title
{SAP} % Organization
{Walldorf, Germany} % Location
{Apr. 2016 - Sept. 2016} % Date(s)
{
  \begin{cvitems} % Description(s) of tasks/responsibilities
    \item {Developed extensions for the MediaWiki knowledge management platform SAPedia.}
    \item {Built and operated applications based on the LAMP stack and Apache Solr.}
    \item {Set up a continuous delivery pipeline using GitHub for Enterprise and Travis CI.}
    \item {Used SAP Monsoon as an IaaS provider to scale the service flexibly and the Chef IaC platform to automate infrastructure deployments.}
    \item {Connected on-premise services with the SAP HANA Cloud Platform using the SAP HANA Cloud Connector.}
    \item {Optimized the Solr search index using Lucene syntax and implemented a custom Solr extension in Java.}
    \item {Implemented applications based on the OpenSocial Gadget Standard and integrated them into the internal SAP ecosystem following SAP Security Procedures using SAML, SSL, PKI, and principal propagation techniques.}
  \end{cvitems}
}

%---------------------------------------------------------
\cventry
{Academic Tutor in Computer Science} % Job title
{Karlsruhe Institute of Technology} % Organization
{Karlsruhe, Germany} % Location
{Oct. 2015 - Sept. 2017} % Date(s)
{
  \begin{cvitems} % Description(s) of tasks/responsibilities
    \item {Wrote sample solutions for the programming lecture students at the Institute for Program Structures and Data Organization.}
    \item {Extended an automatic testing framework for student programming assignment submissions.}
    \item {Moderated the lecture's online question-and-answer platform, ILIAS.}
    \item {Created metamodels for languages conforming to the international standard IEC 61131-3.}
    \item {Designed and implemented a domain-specific language for change propagation analysis using the open-source framework Xtext.}
  \end{cvitems}
}

%---------------------------------------------------------
\cventry
{Software Developer [internship]} % Job title
{PROMATIS software GmbH} % Organization
{Ettlingen, Germany} % Location
{Nov. 2015 - Mar. 2016} % Date(s)
{
  \begin{cvitems} % Description(s) of tasks/responsibilities
     \item {Developed Java software components for the Horus Business Modeler - an Eclipse Rich Client Platform application for BPM modeling.}
     \item {Conducted a runtime performance analysis using JProfiler to speed up the application.}
  \end{cvitems}
}

%---------------------------------------------------------


%---------------------------------------------------------
\end{cventries}

%-------------------------------------------------------------------------------
%	SECTION TITLE
%-------------------------------------------------------------------------------
\cvsection{Honors \& Awards}

%-------------------------------------------------------------------------------
%	SUBSECTION TITLE
%-------------------------------------------------------------------------------
%\cvsubsection{Domestic}


%-------------------------------------------------------------------------------
%	CONTENT
%-------------------------------------------------------------------------------
\begin{cvhonors}

%---------------------------------------------------------
  \cvhonor
    {\href{https://kit-gruenderschmiede.de/gruender-des-monats-nesto/}{EXIST}} % Award
    {\href{https://kit-gruenderschmiede.de/gruender-des-monats-nesto/}{Business Start-up Grant from the German Federal Ministry for Economic Affairs and Energy}} % Event
    {Karlsruhe, Germany} % Location
    {2017} % Date(s)

%---------------------------------------------------------
\end{cvhonors}

%-------------------------------------------------------------------------------
%	SECTION TITLE
%-------------------------------------------------------------------------------
\cvsection{Presentations}


%-------------------------------------------------------------------------------
%	CONTENT
%-------------------------------------------------------------------------------
\begin{cventries}

%---------------------------------------------------------
  \cventry
    {\href{https://www.youtube.com/watch?v=MjBYAA_WZg4&ab_channel=MongoDB}{Speaker for Technical Session <Kosteneffiziente DSGVO Protokollierung at Scale mit Atlas Online Archive>}} % Role
    {\href{https://www.youtube.com/watch?v=MjBYAA_WZg4&ab_channel=MongoDB}{MongoDB.live DACH \entrypositionstyle{[German]}}} % Event
    {Online Event} % Location
    {Dec. 2020} % Date(s)
    {
      \begin{cvitems} % Description(s)
        \item {Presented Nesto's Cloud Architecture, Tech Stack, and Database Cluster.}
        \item {Showed how Nesto saved data storage and backup costs using MongoDB Online Archive.}
        \item {Outlined our approach for SaaS applications to scale cost-effectively while adhering to strict EU GDPR requirements.}
        \item {The story was published by MongoDB as an article on their website: \textit{\href{https://www.mongodb.com/customers/nesto}{Automating hospitality management in uncertain times}}}.
      \end{cvitems}
    }

%---------------------------------------------------------
\end{cventries}

%-------------------------------------------------------------------------------
%	SECTION TITLE
%-------------------------------------------------------------------------------
\cvsection{Certifications}


%-------------------------------------------------------------------------------
%	CONTENT
%-------------------------------------------------------------------------------
\begin{cventries}

%---------------------------------------------------------
  \cventry
    {Amazon Web Services Training and Certification}
    {\href{https://www.youracclaim.com/badges/e00de02f-620b-41b5-8ad7-3caaa7993bc9}{AWS Certified Solutions Architect – Professional}}
    {7C7KWL51FNVQQ9SQ}
    {Jul. 2020 - PRESENT}
    {}

%---------------------------------------------------------
\cventry
    {Amazon Web Services Training and Certification}
    {\href{https://www.youracclaim.com/badges/47430bb9-b9ea-4927-97af-c79b3a9c6d04}{AWS Certified DevOps Engineer – Professional}}
    {512NJSPCB2F11K5J}
    {Jun. 2019 - Jun. 2022}
    {}

%---------------------------------------------------------
\cventry
    {Amazon Web Services Training and Certification}
    {\href{https://www.youracclaim.com/badges/2cd78cd7-189d-41cd-ac59-535b65cef3f9}{AWS Certified Security – Specialty}}
    {T933917CB1F1QLCL}
    {May. 2019 - May. 2022}
    {}

%---------------------------------------------------------
\cventry
    {Amazon Web Services Training and Certification}
    {\href{https://www.youracclaim.com/badges/1eada9e9-bdeb-4bcc-b056-5519c722a312}{AWS Certified Advanced Networking – Specialty}}
    {YJR50BF232QE1QW8}
    {Nov. 2019 - PRESENT}
    {}

%---------------------------------------------------------
\cventry
    {Amazon Web Services Training and Certification}
    {\href{https://www.youracclaim.com/badges/ea88e1a0-800e-4d98-b116-a9614a40966c}{AWS Certified Data Analytics – Specialty}}
    {PCJ3DYYKMJFE1J51}
    {Dec. 2019 - PRESENT}
    {}

%---------------------------------------------------------
\cventry
    {COMPLAVIS Akademie}
    {Certified Data Protection Officer and Trained \& qualified specialist BDSG \& GDPR}
    {022012\#20023830}
    {Mar. 2019 - PRESENT}
    {}

%---------------------------------------------------------
\end{cventries}

%-------------------------------------------------------------------------------
%	SECTION TITLE
%-------------------------------------------------------------------------------
\cvsection{Publications}

%-------------------------------------------------------------------------------
%	SUBSECTION TITLE
%-------------------------------------------------------------------------------

\begin{refsection}
	\nocite{loper2018sprache}
	\nocite{busch2018cross}

	\newrefcontext[sorting=ydnt]
	\printbibliography[heading=none]
\end{refsection}

%-------------------------------------------------------------------------------
%	SECTION TITLE
%-------------------------------------------------------------------------------
\cvsection{Extracurricular Activity}


%-------------------------------------------------------------------------------
%	CONTENT
%-------------------------------------------------------------------------------
\begin{cventries}

%---------------------------------------------------------
  \cventry
    {Amateur soccer player} % Affiliation/role
    {\href{https://www.fupa.net/player/nv-martin-loper-702462}{Post Südstadt Karlsruhe (PSK)}} % Organization/group
    {Karlsruhe, Germany} % Location
    {Okt. 2014 - PRESENT} % Date(s)
    {
      \begin{cvitems} % Description(s) bullet points
        \item {Played as goalkeeper in the Tier VII-Landesliga Mittelbaden (Baden's Football State League) between 2014 and 2016.}
        \item {Played various positions in the Uni-Liga Karlsruhe - the student soccer competition between all higher education institutions in Karlsruhe.}
      \end{cvitems}
    }

%---------------------------------------------------------
\end{cventries}


%-------------------------------------------------------------------------------
\end{document}
